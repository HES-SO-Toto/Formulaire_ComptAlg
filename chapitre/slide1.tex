
\formdesc{Polynôme d'interpolation : }

\formtitle{Forme de lagrange  :} 

\noindent{\hfill$ P_n(x) = \sum_{i=0}^{n} y_i l_i(x) $ \hfill} 
\noindent{\hfill$ l_i = \prod_{j=0,j \neq i}^{n} \cfrac{x -x_j}{x -x_i}  $ \hfill} 

\noindent{\hfill$ L(x) = \prod_{i=0}^{n} (x -x_i) $ \hfill} 
\noindent{\hfill$ \lambda_i^{-1} = \prod_{j=0,j \neq i}^{n} (x_i - x_j) $ \hfill} 

\formtitle{Evaluation :} 

\noindent{ \hfill$P_n =  L(x) \sum_{i=0}^{n} \cfrac{\lambda_i}{x-x_i} y_i $ \hfill} 

\formtitle{Formule barycentrique :} 

\noindent{ \hfill$P_n =  \cfrac{\sum_{i=0}^{n} \cfrac{\lambda_i}{x-x_i} y_i }{\sum_{i=0}^{n} \cfrac{\lambda_i}{x-x_i}}$ \hfill} 

\formtitle{forme de Newton :} 

\noindent{ \hfill$f[x_0] = y_0 = c_0$ \hfill}
\noindent{ \hfill$f[x_n] = y_n$ \hfill}

\noindent{ \hfill$f[x_0,x_1] = \cfrac{y_1 - y_0}{x_1 - x_0}$ \hfill}

\noindent{ \hfill$f[x_0,x_1,x_2] = \cfrac{f[x_1,x_2] - f[x_0,x_1]}{x_2 - x_0}$ \hfill}

\noindent{ \hfill$f[x_0,x_1,...,x_n] = \cfrac{f[x_1,x_2,...,x_n] - f[x_0,x_1,...,x_{n-1}]}{x_n - x_0}$ \hfill}

\formtitle{Schéma de newtown:} 

{
\hspace{-1cm}
\begin{tikzpicture}
    \footnotesize
    % Nodes
    \node (x0) at (0,0) {$x_0$};
    \node[right=0.5cm of x0] (f_x0) {$f[x_0] = c_0$};
  
    \node[below=0.5cm of x0] (x1) {$x_1$};
    \node[right=0.5cm of x1] (f_x1) {$f[x_1]$};
    \node[below right=-.1cm and 1cm of f_x0] (f_x0_x1) {$f[x_0, x_1] = c_1$};

    \node[below=.5cm of x1] (x2) {$x_2$};
    \node[right=0.5cm of x2] (f_x2) {$f[x_2]$};
    \node[below=.5cm of f_x0_x1] (f_x1_x2) {$f[x_1, x_2]$};
    \node[below right=-.1cm and 1cm of f_x0_x1] (f_x0_x1_x2) {$f[x_0, x_1, x_2] = c_2$};
   

    \draw (x0.east) ++(0.25,0.25) -- ++(0,-2.5);

    % Arrows
    \draw[->,>=Stealth] (f_x0) -- node[above] {-} (f_x0_x1);
    \draw[->,>=Stealth] (f_x1) -- node[below] {+} (f_x0_x1);
    \draw[->,>=Stealth] (f_x1) -- node[below] {-} (f_x1_x2);
    \draw[->,>=Stealth] (f_x2) -- node[below] {+} (f_x1_x2);

    \draw[->,>=Stealth] (f_x0_x1) -- node[above] {-} (f_x0_x1_x2);
    \draw[->,>=Stealth] (f_x1_x2) -- node[below] {+} (f_x0_x1_x2);

  \end{tikzpicture}


  \noindent{ \hfill$P_n(x) = c_0 + c_n \prod_{i=0}^{n-1}(x - x_i) $ \hfill}
}

\formtitle{Schéma de horner:}

$P_n = a\cdot x^N + b\cdot x^(N-1) + ... + e$


\begin{tikzpicture}
  \usetikzlibrary{positioning}

  % Nodes
  \node (x4) at (1,0) {$x^4$};
  \node[right=1cm of x4] (x3) {$x^3$};
  \node[right=1cm of x3] (x2) {$x^2$};
  \node[right=1cm of x2] (x1) {$x^1$};
  \node[right=1cm of x1] (x0) {$x^0$};
  \node[below=0.15cm of x4] (a) {$a$};
  \node[below=0.15cm of x3] (b) {$b$};
  \node[below=0.15cm of x2] (c) {$c$};
  \node[below=0.15cm of x1] (d) {$d$};
  \node[below=0.15cm of x0] (e) {$e$};
  %line 
  \draw (a.south) ++(-1cm,-0.15) -- ++(8,0);
  \node[below=1.25cm of a] (A) {$A$};
  \node[left=0.25cm of A] (xe) {$x_{e}$};
  \node[right=1cm of A] (B) {$B$};
  \node[right=1cm of B] (C) {$C$};
  \node[right=1cm of C] (D) {$D$};
  \node[right=1cm of D] (Result) {$P(x_{e})$};


  \node[above=0.5cm of B] (AEVAL) {$A_{Eval}$};
  \node[above=0.5cm of C] (BEVAL) {$B_{Eval}$};
  \node[above=0.5cm of D] (CEVAL) {$C_{Eval}$};
  \node[above=0.5cm of Result] (DEVAL) {$D_{Eval}$};
  
  \draw (A.south) ++(-1cm,-0.15) -- ++(8,0);

  \node[below=1.25cm of A] (A') {$A'$};
  \node[left=0.25cm of A'] (xe) {$x_{e}$};
  \node[right=1cm of A'] (B') {$B'$};
  \node[right=1cm of B'] (C') {$C'$};
  \node[right=1cm of C'] (Result') {$P'(x_{e})$};

  \node[above=0.5cm of B'] (AEVAL') {$A_{Eval}'$};
  \node[above=0.5cm of C'] (BEVAL') {$B_{Eval}'$};
  \node[above=0.5cm of Result'] (CEVAL') {$C_{Eval}'$};

  \node[above=0.5cm of B] (AEVAL) {$A_{Eval}$};
  \node[above=0.5cm of C] (BEVAL) {$B_{Eval}$};
  \node[above=0.5cm of D] (CEVAL) {$C_{Eval}$};
  \node[above=0.5cm of Result] (DEVAL) {$D_{Eval}$};


  % Arrows
  \draw[->,>=Stealth] (a.south) -- node[above] {} (A.north);
  \draw[->,>=Stealth,color = Green] (A.east) -- node[midway,right] {$\cdot x_{e}$} (AEVAL.west);
  \draw[->,>=Stealth,color =Blue] (AEVAL.south) -- node[midway,right] {$+b$} (B.north);
  \draw[->,>=Stealth,color = Green] (B.east) -- node[midway,right] {$\cdot x_{e}$} (BEVAL.west);
  \draw[->,>=Stealth,color =Blue] (BEVAL.south) -- node[midway,right] {$+c$} (C.north);
  \draw[->,>=Stealth,color = Green] (C.east) -- node[midway,right] {$\cdot x_{e}$} (CEVAL.west);
  \draw[->,>=Stealth,color =Blue] (CEVAL.south) -- node[midway,right] {$+d$} (D.north);
  \draw[->,>=Stealth,color = Green] (D.east) -- node[midway,right] {$\cdot x_{e}$} (DEVAL.west);
  \draw[->,>=Stealth,color =Blue] (DEVAL.south) -- node[midway,right] {$+e$} (Result.north);

  \draw[->,>=Stealth] (A.south) -- node[midway,right] {} (A'.north);
  \draw[->,>=Stealth,color = Green] (A'.east) -- node[midway,right] {$\cdot x_{e}$} (AEVAL'.west);
  \draw[->,>=Stealth,color =Blue] (AEVAL'.south) -- node[midway,right] {$+b$} (B'.north);
  \draw[->,>=Stealth,color = Green] (B'.east) -- node[midway,right] {$\cdot x_{e}$} (BEVAL'.west);
  \draw[->,>=Stealth,color =Blue] (BEVAL'.south) -- node[midway,right] {$+c$} (C'.north);
  \draw[->,>=Stealth,color = Green] (C'.east) -- node[midway,right] {$\cdot x_{e}$} (CEVAL'.west);
  \draw[->,>=Stealth,color =Blue] (CEVAL'.south) -- node[midway,right] {$+d$} (Result'.north);

\end{tikzpicture}

$P_n = E + x\cdot (D + x\cdot (C + x\cdot (B + x\cdot A)))$

\formtitle{Erreur d'interpolation :} 

\noindent{ \hfill$ \left\lvert f(x) - P_n(x) \right\rvert \leq \cfrac{M_{n+1}}{(n+1)!} \left\lvert \prod_{i=0}^{n}(x-x_i) \right\rvert $ \hfill}

\noindent{ \hfill$ M_{n+1} := max_{\xi \in [a,b]} \left\lvert f^{(n+1)}(\xi)\right\rvert  $ \hfill}
\newpage


\formdesc{Splines : }

\formtitle{Calcul d’une spline cubique :} 

\noindent{ \hfill$S_i(x) = a_i(x-x_i)^3 + b_i(x-x_i)^2 + c_i(x-x_i) + d_i$ \hfill}

\noindent{ \hfill$a_i = \cfrac{1}{6h}(y''_{i+1} - y''_i)$ \hfill}
\noindent{ \hfill$b_i = \cfrac{1}{2}(y''_i)$ \hfill}


\noindent{ \hfill$c_i = \cfrac{y_{i+1} - y_i}{h} - \cfrac{h}{6}(y''_{i+1} + 2y''_i)$ \hfill}
\noindent{ \hfill$d_i = y_i$ \hfill}

\formtitle{Calcul des dérivées secondes :}

\noindent{ \hfill$y''_{i-1} + 4y''_i + y''_{i+1} = \cfrac{6}{h^2}(y_{i+1} - 2y_i + y_{i-1})$ \hfill}

avec $i = 1, ..., n-1$

\formtitle{Matrice tridiagonale :}

\[
\begin{bmatrix}
4 & 1 & 0 & 0 \\
1 & 4 & 1 & 0 \\
0 & 1 & 4 & 1 \\
0 & 0 & 1 & 4 \\
\end{bmatrix}
\begin{bmatrix}
y''_1 \\
y''_2 \\
y''_3 \\
y''_4 \\
\end{bmatrix}
=
\cfrac{6}{h^2}
\begin{bmatrix}
y_2 - 2y_1 + y_0 \\
y_3 - 2y_2 + y_1 \\
y_4 - 2y_3 + y_2 \\
y_5 - 2y_4 + y_3 \\
\end{bmatrix}
-
\begin{bmatrix}
y''_0 \\
0 \\
0 \\
y''_5 \\
\end{bmatrix}
\]
Sur la calculatrice : 
\begin{itemize}
    \item Matrice tridiagonale sto->(ctrl + var) -> A
    \item menu -> Matrice et vecteur(7) -> simultané(6)
    \item A,vecteur colonne de droite -> :=
\end{itemize}


