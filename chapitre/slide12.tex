\hformbar{}

\formtitle{Resolition numerique des EDOs}

\formdesc{modéle exponentiel}

\begin{align*}
    \frac{d\vec{x}}{dt} &= \lambda \vec{x} \\
    \vec{x}(0) &= \vec{x}_0
\end{align*}

\formdesc{modéle logistique}

\begin{align*}
    \frac{dx}{dt} &= k x(t) (M-x(t)) \\
    x(0) &= x_0
\end{align*}

\formdesc{modéle de Lotka-Volterra}

\begin{align*}
    \frac{dx}{dt} &= \alpha x(t) - \beta x(t)y(t) \\
    \frac{dy}{dt} &= \gamma x(t)y(t) - \delta y(t) \\
    x(0) &= x_0 \\
    y(0) &= y_0
\end{align*}

\formdesc{Problème de Cauchy}

\begin{align*}
    \frac{d\vec{u}}{dt} &= \vec{f}(t, \vec{u}(t)) \\
    \vec{u}(t_0) &= \vec{u}_0
\end{align*}

Si ED n'est pas sous la forme de Cauchy, on peut faire une reduction de l'ordre.

Transformation de l'EDO d'ordre $n \leq 1$

$$ \frac{d^ny}{dt^n} = y^{(n)} = f(t, y, y', y'', \dots,y^{(n-1)} = f(t, y_1, y_2, \dots,y_{n(t)}) $$

$$\cfrac{d\vec{u}(t)}{dt} = \begin{bmatrix}
    y1(t) \\
    y2(t) \\
    \vdots \\
    y_{n-1}(t) \\
    y_n(t)
\end{bmatrix} = \begin{bmatrix}
    y2(t) \\
    y3(t) \\
    \vdots \\
    y_n(t) \\
    f(t, y_1, y_2, \dots,y_{n(t)})
\end{bmatrix} = \vec{f}(t, \vec{u}(t))$$

+ condition initiale $\vec{u}(t_0) = \vec{u}_0 = \begin{bmatrix}
    y_1(t_0) \\
    y_2(t_0) \\
    \vdots \\
    y_n(t_0)
\end{bmatrix}$

\formdesc{Méthode d'Euler explicite}

$$ \cfrac{du}{dt} = f(t, u(t)) \quad u(t_0) = u_0 $$

Discrétisation de l'intervalle $[t_0, t_0 + T]$ en $N$ sous-intervalles de longueur $h = \cfrac{T}{N}$

$$ t_{k+1} = t_k + h \quad k = 0, 1, \dots, N-1 $$

$$ u_{k+1} = u_k + h f(t_k, u_k) \quad k = 0, 1, \dots, N-1 $$

\formdesc{Méthode d'Euler modifiée (Heun)}

$$ \vec{s_1} = \vec{f}(t_k, \vec{u}_k) $$
$$ \vec{s_2} = \vec{f}(t_k + h, \vec{u}_k + h \vec{s_1}) $$
$$ \vec{u}_{k+1} = \vec{u}_k + \cfrac{h}{2} (\vec{s_1} + \vec{s_2}) $$

\formdesc{Méthode de point milieu}

$$ \vec{s_1} = \vec{f}(t_k, \vec{u}_k) $$
$$ \vec{s_2} = \vec{f}(t_k + \cfrac{h}{2}, \vec{u}_k + \cfrac{h}{2} \vec{s_1}) $$
$$ \vec{u}_{k+1} = \vec{u}_k + h \vec{s_2} $$

\formdesc{Méthode de optimale}

$$ \vec{s_1} = \vec{f}(t_k, \vec{u}_k) $$
$$ \vec{s_2} = \vec{f}(t_k + \cfrac{h}{2}, \vec{u}_k + \cfrac{h}{2} \vec{s_1}) $$

$$ \vec{u}_{k+1} = \vec{u}_k + h (\cfrac{1}{4} \vec{s_1} + \cfrac{3}{4} \vec{s_2}) $$

\formdesc{Méthode de Runge-Kutta à 2 étages}

$$ \vec{s_1} = \vec{f}(t_k, \vec{u}_k) $$
$$ \vec{s_2} = \vec{f}(t_k + c_2 h, \vec{u}_k + a_{21} h \vec{s_1}) $$
$$ \vec{u}_{k+1} = \vec{u}_k + h (w_1 \vec{s_1} + w_2 \vec{s_2}) $$

$$ w_1 = 1 - \cfrac{1}{2c_2} \quad w_2 = \cfrac{1}{2c_2} $$
\formdesc{Tableau de Butcher}

\begin{tabular}{c|cc}
    $0$ &  &  \\
    $c_2$ & $a_{21}$ &  \\
    \hline
          & $w_1$    & $w_2$ \\
\end{tabular}


\formdesc{Méthode de Runge-Kutta à 3 étages}

$$ \vec{s_1} = \vec{f}(t_k, \vec{u}_k) $$
$$ \vec{s_2} = \vec{f}(t_k + c_2 h, \vec{u}_k + a_{21} h \vec{s_1}) $$
$$ \vec{s_3} = \vec{f}(t_k + c_3 h, \vec{u}_k + a_{31} h \vec{s_1} + a_{32} h \vec{s_2}) $$
$$ \vec{u}_{k+1} = \vec{u}_k + h (w_1 \vec{s_1} + w_2 \vec{s_2} + w_3 \vec{s_3}) $$

$$ w_1 = 1 - \cfrac{1}{2c_2} - \cfrac{1}{2c_3} \quad w_2 = \cfrac{1}{2c_2} \quad w_3 = \cfrac{1}{2c_3} $$

\begin{tabular}{c|ccc}
    $0$ &  &  &  \\
    $c_2$ & $a_{21}$ &  &  \\
    $c_3$ & $a_{31}$ & $a_{32}$ &  \\
    \hline
          & $w_1$    & $w_2$    & $w_3$ \\
\end{tabular}

\formdesc{Méthode de Runge-Kutta à 4 étages}

$$ \vec{s_1} = \vec{f}(t_k, \vec{u}_k) $$
$$ \vec{s_2} = \vec{f}(t_k + c_2 h, \vec{u}_k + a_{21} h \vec{s_1}) $$
$$ \vec{s_3} = \vec{f}(t_k + c_3 h, \vec{u}_k + a_{31} h \vec{s_1} + a_{32} h \vec{s_2}) $$
$$ \vec{s_4} = \vec{f}(t_k + c_4 h, \vec{u}_k + a_{41} h \vec{s_1} + a_{42} h \vec{s_2} + a_{43} h \vec{s_3}) $$
$$ \vec{u}_{k+1} = \vec{u}_k + h (w_1 \vec{s_1} + w_2 \vec{s_2} + w_3 \vec{s_3} + w_4 \vec{s_4}) $$

\begin{tabular}{c|cccc}
    $0$ &  &  &  &  \\
    $\cfrac{1}{2} $ & $\cfrac{1}{2}$ &  &  &  \\
    $\cfrac{1}{2} $ & $0$ & $\cfrac{1}{2}$ &  &  \\
    $1$ & $0$ & $0$ & $1$ &  \\
    \hline
          & $1/6$    & $1/3$    & $1/3$ & $1/6$ \\
\end{tabular}

\formdesc{Estimation de l'erreur}

$$ \vec{u}_{k+1} = \vec{u}_k + h \Phi(t_k, \vec{u}_k, h) $$

Type d'erreur :

\begin{itemize}
    \item Globale : commis dans l'intervalle $[t_0, t_0 + T]$
    
    $E(T) = \vec{u}(T) - \vec{u}_n$
    \item locale : commis à chaque étape $k$
    \item 
    $e_{k+1} = \vec{u}(t_{k+1}) - \vec{u}_{k+1} = \vec{u}(t_{k+1}) - \vec{u}(t_k) - h \Phi(t_k, \vec{u}_k, h)$
    \item arrondi : approximations numériques des calculs
    
    $R_{k+1} = \vec{u}_{k+1} - \vec{U}_{k+1}$

    \item total : $|u(T) - U_n| \leq |E(T)| + |R_n|$

\end{itemize}