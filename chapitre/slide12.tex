\hformbar{}

\formtitle{Resolition numerique des EDOs}

\formdesc{modéle exponentiel}

\begin{align*}
    \frac{d\vec{x}}{dt} &= \lambda \vec{x} \\
    \vec{x}(0) &= \vec{x}_0
\end{align*}

\formdesc{modéle logistique}

\begin{align*}
    \frac{dx}{dt} &= k x(t) (M-x(t)) \\
    x(0) &= x_0
\end{align*}

\formdesc{modéle de Lotka-Volterra}

\begin{align*}
    \frac{dx}{dt} &= \alpha x(t) - \beta x(t)y(t) \\
    \frac{dy}{dt} &= \gamma x(t)y(t) - \delta y(t) \\
    x(0) &= x_0 \\
    y(0) &= y_0
\end{align*}

\formdesc{Problème de Cauchy}

\begin{align*}
    \frac{d\vec{u}}{dt} &= \vec{f}(t, \vec{u}(t)) \\
    \vec{u}(t_0) &= \vec{u}_0
\end{align*}

Si ED n'est pas sous la forme de Cauchy, on peut faire une reduction de l'ordre.

Transformation de l'EDO d'ordre $n \leq 1$

$$ \frac{d^ny}{dt^n} = y^{(n)} = f(t, y, y', y'', \dots,y^{(n-1)} = f(t, y_1, y_2, \dots,y_{n(t)}) $$

$$\cfrac{d\vec{u}(t)}{dt} = \begin{bmatrix}
    y1(t) \\
    y2(t) \\
    \vdots \\
    y_{n-1}(t) \\
    y_n(t)
\end{bmatrix} = \begin{bmatrix}
    y2(t) \\
    y3(t) \\
    \vdots \\
    y_n(t) \\
    f(t, y_1, y_2, \dots,y_{n(t)})
\end{bmatrix} = \vec{f}(t, \vec{u}(t))$$

+ condition initiale $\vec{u}(t_0) = \vec{u}_0 = \begin{bmatrix}
    y_1(t_0) \\
    y_2(t_0) \\
    \vdots \\
    y_n(t_0)
\end{bmatrix}$

