\hformbar

\formdesc{Intégration : }

\formtitle{Quadrature interpolation :}

\formtitle{formule interpolatoire :}

$$Q_n = \int_{a}^{b} P_n(x) dx = \sum_{i=0}^{n} f(x_k) \int_{a}^{b} L_k(x) dx = \sum_{i=0}^{n} f(x_k) w_k$$

avec : $$w_k = \int_{a}^{b} L_k(x) dx = \int_{a}^{b} \prod_{j=0, j \neq k}^{n} \frac{x-x_j}{x_k-x_j} dx$$

\begin{itemize}
    \item $Q_n$ est une formule d'intégration de degré au moins $n$
    \item $x_k$ sont les points de quadrature
    \item $w_k$ sont les poids de quadrature
\end{itemize}

$$ E[f] = Q-I = \sum_{i=0}^{n} f(x_k) w_k - \int_{a}^{b} f(x) dx$$

\hformbar

\formdesc{Formule de Newton-Cotes :}

Les formules interpolatoires polynomiales avec noeuds
équidistants sont appelées des formules de Newton-Cotes

\formtitle{Fermées (car a et b en font partie) :}

$$x = a + (b-a)t \quad t \in [0,1] \Leftrightarrow dx = (b-a)dt$$

$$w_k = (b-a) \int_{0}^{1} \prod_{j=0, j \neq k}^{n} \frac{nt-j}{k-j} dt$$

$$\quad f_0 = f(a)$$

\formtitle{ Trapezes (degré de precision 1) :}

$Q_1 = \frac{h}{2} (f_0 + f_1) \quad h = b-a$

\formtitle{ Simpson (degré de precision 3) :}

$Q_2 = \frac{h}{3} (f_0 + 4f_1 + f_2) \quad h = \cfrac{b-a}{2} $

\formtitle{ 3/8 de Newtown (degré de precision 3) :}

$Q_3 = \frac{3h}{8} (f_0 + 3f_1 + 3f_2 + f_3) \quad h = \cfrac{b-a}{3} \quad f_0 = f(a)$

\formtitle{ Boole (degré de precision 5) :}

$Q_4 = \frac{2h}{45} (7f_0 + 32f_1 + 12f_2 + 32f_3 + 7f_4) \quad h = \cfrac{b-a}{4}$

\formtitle{ degré de precision :}
\begin{itemize}
    \item Prendre nombre de point impaire.
    \item subdiviser l'intervalle en sous intervalle de taille égale.
\end{itemize}


\formtitle{ Regles des trapezes composées :}

$$T(h) = h\left[ \cfrac{1}{2} f(a) + \sum_{i=1}^{n-1} f(x_i) + \cfrac{1}{2} f(b) \right]$$

avec $$h = \cfrac{b-a}{n}$$


\formtitle{ Regle du point milieu}

$$M(h) = h \sum_{i=1}^{n} f(x_{k+\frac{1}{2}}) \quad x_{k+\frac{1}{2}} = a + (k + \frac{1}{2})h$$

Lien avec la regle des trapezes : 

$$T\left(\cfrac{h}{2}\right) = \cfrac{1}{2} \left[T(h) + M(h)\right]$$

exemple : 

\begin{itemize}
    \item $T(h) = h/2 (f_0+f_4)$
    \item M(h) = $h(f_2)$
    \item $T(h/2) = 1/2 (T(h) + M(h))$
    \item $M(h/2) = h/2 (f_1+f_3)$
    \item $T(h/4) = 1/2 (T(h/2) + M(h/2))$
\end{itemize}

\formtitle{Erreur (trapezes composées) :}
convergence lente sauf cas optimal :
\begin{itemize}
    \item f(x) est périodique
    \item f(x) infiniement dérivable
    \item Intergrale sur une période
\end{itemize}

$$\left\lvert \int_{a}^{b} f(x) dx - T(h) \right\rvert \leq \cfrac{h^2 (b-a)}{12} max_{x \in [a,b]} \left\lvert f''(x) \right\rvert$$
\formtitle{ Regles des Simpson composées :}

Sous 4 interval de taille h = (b-a)/4

$$S_c = \cfrac{h}{3} \left[ f_0 + 4f_1 + 2f_2 + 4f_3 + f_4 \right]$$

\formtitle{ Paire : }
2n sous intervalle de taille h = (b-a)/2n

$$S_c = \cfrac{h}{3} \left[ f(a)+4f(x_1) + f(b) + 2 \sum_{i=1}^{n-1} f(x_{2i}) + 2f(x_{2i+1}) \right]$$ 

\formtitle{ Erreur (Simpson composées) :}

$$\left\lvert \int_{a}^{b} f(x) dx - S_c \right\rvert \leq \cfrac{h^4 (b-a)}{180} max_{x \in [a,b]} \left\lvert f^{(4)}(x) \right\rvert$$




