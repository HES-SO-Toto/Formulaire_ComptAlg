\hformbar{}

\formdesc{Courbes de Bézier :}

\formtitle{Algorithme de De Casteljau :}

\formtitle{Ordre 1 :}

$P_0 = \binom{1}{1} \quad P_1 = \binom{2}{3}$

alors 

$P(t) = (1-t)P_0 + tP_1 = \binom{1}{1}(1-t) + \binom{2}{3}t = \binom{1-t}{1+2t}$

\formtitle{Ordre 2 :}

On construit d'abord deux courbes de Bézier d'ordre 1

$P_0 = \binom{1}{1} \quad P_1 = \binom{2}{3} \quad P_2 = \binom{4}{2}$

\begin{align*}
P(t) &= (1-t)Q_0(t) + tQ_1(t) \\
& = (1-t)((1-t)P_0 + tP_1) + t((1-t)P_1 + tP_2) \\
& = (1-t)^2P_0 + 2t(1-t)P_1 + t^2P_2 \\
\end{align*}

\formtitle{Ordre 3 :}

On construit d'abord deux courbes de Bézier d'ordre 2   

\begin{align*}
    P(t) &= (1-t)R_0(t) + tR_1(t) \\
    & = \dots \\
    & = (1-t)^3P_0 + 3t(1-t)^2P_1 + 3t^2(1-t)P_2 + t^3P_3 \\
\end{align*}

\formtitle{Propriétés :}

\begin{itemize}
    \item points initialement donnés = points de contrôle
    \item Les courbes de Bézier passent par le premier et le dernier point de contrôle
    \item Les courbes de Bézier sont contenues dans l'enveloppe convexe des points de contrôle
    \item Les courbes de Bézier sont stables sous transformations affines (translation, rotation, homothétie)
\end{itemize}

\formtitle{Polynôme de Bernstein :}

$$B_i^n(t) = \binom{n}{i}t^i(1-t)^{n-i} \quad i = 0,1, \dots, n$$

\formtitle{Relation avec les courbes de Bézier :}

$$P(t) = \sum_{i=0}^{n} B_i^n(t)P_i$$

\formtitle{Propriétés :}

$$ P'(t) = n \sum_{i=0}^{n-1} (P_{i+1} - P_i) B_i^{n-1}(t)$$

en t = 0 et t = 1 :

$$ P'(0) = n(P_1 - P_0) \quad P'(1) = n(P_n - P_{n-1})$$

la direction de la tangente en $P_0$ est identique à celle de la droite $P_0P_1$
et la direction de la tangente en $P_n$ est identique à celle de la droite $P_{n-1}P_n$