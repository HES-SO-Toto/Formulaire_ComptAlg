
\hformbar{}

\formdesc{ Résolition des systèmes linéaires}

resoudre $A\vec{x} = \vec{b}$ pour trouver $\vec{x} \in \mathbb{R}^n$.

\formtitle{hypotheses}

\begin{itemize}
    \item $A$ est une matrice $n \times n$.
    \item $A$ est inversible.
\end{itemize}

\formtitle{methodes}

\begin{itemize}
\item \textbf{méthode directe}: $A^{-1}$ est calculé explicitement.

Exemple : Elimination de Gauss, $LU$, $LL^t$, $QR$

\item \textbf{méthode iterative}: $\vec{x}$ est calculé par une suite de vecteurs $\vec{x}_0, \vec{x}_1, \vec{x}_2, \dots$.
\end{itemize}

\formtitle{méthode iterative}

\formtitle{Normes vecteurs}

Classique : 

$$\left||v|\right|_p = \left(\sum_{i=1}^n \left|v_i\right|^p\right)^{1/p}$$

\begin{itemize}
    \item $p=1$: norme $L_1$ (norme de Manhattan)
    
    $$\left||v|\right|_1 = \sum_{i=1}^n \left|v_i\right|$$
    \item $p=2$: norme $L_2$ (norme euclidienne)
    
    $$\left||v|\right|_2 = \sqrt{\sum_{i=1}^n \left|v_i\right|^2}$$
    \item $p=\infty$: norme $L_\infty$ (norme du maximum)
    
    $$\left||v|\right|_\infty = \max_{1 \leq i \leq n} \left|v_i\right|$$
\end{itemize}

\formtitle{Normes matricielles}

$$\left||A\vec{v}|\right|_p \leq \left||A|\right|_p \left||\vec{v}|\right|_p \quad \forall \vec{v} \in \mathbb{R}^n \Longleftrightarrow \left||A|\right|_p = \max_{\vec{v} \neq 0} \frac{\left||A\vec{v}|\right|_p}{\left||\vec{v}|\right|_p}$$

\begin{itemize}
    \item $p=1$: norme $L_1$ 
    $$\left||A|\right|_1 = \max_{j} \sum_{i=1}^n \left|a_{ij}\right|$$
    \item $p=2$: norme $L_2$ 
    $$\left||A|\right|_2 = max |\lambda(A^TA)|$$
    \item $p=\infty$: norme $L_\infty$ 
    $$\left||A|\right|_\infty = \left||A^T|\right|_1 =\max_{i} \sum_{j=1}^n \left|a_{ij}\right|$$
\end{itemize}

\formtitle{Lien entre $||\vec{x} - \vec{x}_k||$ et $||\vec{b} - A\vec{x}_{k}||$}

\begin{align*}
    \underbrace{\cfrac{||\vec{x} - \vec{x}_k||_p}{||\vec{x}||_p}} &\leq \underbrace{ ||A||_p ||A^{-1}||_p} & \underbrace{\cfrac{||\vec{b} - A\vec{x}_{k}||_p}{||\vec{b}||_p}} \\
    \text{erreur relative} & \leq \kappa_p(A)  & \text{résidu relatif}
\end{align*}

$\kappa_p(A)$ est le conditionnement de $A$.
Si $\kappa_p(A)$ est grand, alors $A$ est mal conditionné.

