\formdesc{Methode directe}

\formtitle{Elimination de Gauss}

\begin{enumerate}
    \item Transformer la matrice $A$ en forme echelonnee.
    \item Substituer les valeurs trouvees dans les equations.
\end{enumerate}

\formtitle{Factorisation LU}

\begin{enumerate}
    \item Transformer la matrice $A$ en forme echelonnee. $ A = LU $
    \item Substituer en avant $ L\vec{y} = \vec{b} $. avec L la matrice triangulaire inferieure.
    \item Substituer en arriere $ U\vec{x} = \vec{y} $. avec U la matrice triangulaire superieure.
\end{enumerate}

voir slide 9.

\formtitle{méthode de Doolittle}

$$u_{km} = a_{km} - \sum_{j=1}^{k-1} l_{kj}u_{jm} \quad m = k, k+1, \dots, n $$

$$l_{ik} = \frac{1}{u_{kk}} \left( a_{ik} - \sum_{j=1}^{k-1} l_{ij}u_{jk} \right) \quad m = k+1, k+2, \dots, n$$

\formtitle{Besoin de pivotement}

Elim. de Gauss/LU sans pivotement peut echouer si $a_{kk} = 0$.

Permuter les lignes de $A$ pour avoir $a_{kk} \neq 0$.
Mettre le plus grand element en valeur absolue en haut.

$$LU= PA$$

P est une matrice de permutation.

Initialiser $P$ comme la matrice identite I.

Si $A$ subit une permutation, $P$ doit subir la meme permutation.
